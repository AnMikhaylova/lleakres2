% введение

Анализ данных – это комплекс методов и приложений, связанных с алгоритмами обработки данных и не имеющих четко фиксированного ответа на каждый поступающий объект. Это будет выделять их от классических алгоритмов, например реализующих сортировку или словарь. Статистическая обработка данных и визуализация результатов анализа — это неизбежный шаг работы с данными, полученными в различных областях естественных наук, в социологии, психологии или экономике.

Необходимым условием современного статистического анализа данных является эффективное использование компьютерных программ, от функциональной полноты и алгоритмической продуманности которых зависит итоговая интерпретация результатов исследования и надежность выводов. 

Вычислительные программы, используемые для моделирования сложных физических процессов, зачастую имеют значительное количество настроечных параметров и условий выполнения расчетов. Это относится и к программе blleak16d, параметры которой хранятся в текстовых файлах и редактируются непосредственно в них. Помимо конфигурационных файлов, существуют так же расчетные файлы, содержащие информацию об итогах расчета вычислительной программы. Такие файлы перенасыщенны информацией причем очень часто повторяющейся, что усложняет процесс анализа. Чтобы сделать грамотный анализ расчётных данных и их визуализацию, нужно выбрать различные параметры и переменные для отображения. С учётом объёма информации в файлах, сделать это рациональнее с помощью компьютерной программы с графическим интерфейсом.  


Целью данной научно-исследовательской работы является разработка Java-приложения с графическим интерфейсом для задания параметров визуализации. 
\medskip

В ходе выполнения работы необходимо решить следующие задачи:

\begin{itemize}
	\item выполнить анализ структуры конфигурационных и расчётных файлов;
	\item разработать вспомогательные классы для считывания конфигурации;
	\item разработать классы, описывабщие структуру расчетных файлов;
	\item спроектировать и разработать пользовательский интерфейс задания параметров расчётных данных.	
\end{itemize}